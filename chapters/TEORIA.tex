
De acordo com a estrutura do trabalho proposto, o embasamento teórico desenvolvido busca mostrar os passos para o projeto do inversor, do conversor  elevador e o modelo de turbina e gerador síncrono de imãs permanentes (PMSG). O modelo da rede e carga são apresentados na simulação e buscam representar o comportamento normal da rede elétrica monofásica de distribuição.

Os modelos desenvolvidos foram:

\begin{itemize}
    \item Inversor monofásico de ponte completa;
    \item Conversor elevador CC/CC tipo \textit{boost};
    \item Turbina eólica de pequeno porte sem controle de ângulo (\textit{pitch});
    \item Gerador síncrono de imãs permanentes (\textit{Permanent Magnet Synchronous Generator - PMSG});
\end{itemize}

Com o objetivo de controlar a velocidade angular da turbina, foi desenvolvido um controlador MPPT para o conjunto de geração. Este controle atua na corrente do conversor aumentando ou diminuindo a carga vista pelos terminais do gerador. Desta forma, o torque eletromagnético de oposição imposto pelo gerador aumenta ou diminui controlando a velocidade angular da turbina. Assim, seguindo as curvas de potência versus velocidade angular, é possível obter o ponto de operação que maximiza a extração de potência para cada velocidade de vento.




