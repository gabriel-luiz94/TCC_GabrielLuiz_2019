
De acordo com a estrutura do trabalho proposto, o embasamento teórico desenvolvido busca mostrar os passos para o projeto do inversor, do conversor cc/cc boost e o modelo de turbina e gerador síncrono de imãs permanentes (PMSG). O modelo da rede e carga são apresentados na simulação e buscam representar o comportamento normal da rede elétrica de distribuição.

Desta forma os modelos desenvolvidos foram:

\begin{itemize}
    \item Inversor monofásico de ponte completa;
    \item Conversor elevador CC/CC tipo \textit{boost};
    \item Turbina eólica de pequeno porte sem controle de ângulo (\textit{pitch});
    \item Gerador síncrono de imãs permanentes (\textit{Permanent Magnet Synchronous Generator - PMSG});
\end{itemize}


De acordo com a topologia adotada neste trabalho, a tensão de saída do gerador apresenta nível de tensão e frequência variável. Desta forma, não é possível a conexão direta a rede. Além disso não há mecanismos que garantam o melhor aproveitamento da energia disponível o que não é desejável em geração distribuída. 

Assim utilizou-se um conversor CC/CC e um inversor para controlar a tensão e frequência de saída do sistema garantindo a sincronização com a rede. Além disso, buscou-se controlar a velocidade da turbina através de um controle MPPT (Maximum Power Point Tracking)
\section{Inversor Monofásico}
