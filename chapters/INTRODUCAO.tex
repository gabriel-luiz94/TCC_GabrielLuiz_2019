% Gabriel Luiz De França Sales
% Trabalho de Conclusão de Curso
% Tema: Estudo de Aerogerador Síncrono de Imã Permanente Trifásico com e sem Controle MPPT Conectado à Rede Elétrica Monofásica
%
% Introdução
% Neste arquivo será apresentado o trabalho e sua motivação, além disso serão explicadas as metodologias utilizadas e a estrutura do trabalho

% 1 - Introdução



% 1.1 - Motivação
\section{Motivação}  


% 1.2 - Objetivos
\section{Objetivos}

O objetivo deste trabalho é avaliar o funcionamento de um aerogerador síncrono de imã permanente com e sem controle de velocidade baseado no rastreamento do ponto de máxima potência - MPPT. Desta forma este trabalho visa implementar, a partir de modelagem computacional, o controle de velocidade da turbina assim como o sistema de conexão com a rede elétrica monofásica. Assim, torna-se necessário o estudo dos componentes necessários ao sistema e o dimensionamento adequado dos controladores e filtros utilizados.


% 1.2.1 - Objetivos específicos
\subsection{Objetivos específicos}

Neste trabalho pretende-se desenvolver respectivamente:

\begin{itemize}
	\item Projeto e utilização de um inversor monofásico para a conexão com a rede elétrica juntamente com o controle de tensão do barramento CC à entrada do inversor e o controle de corrente em fase com a tensão da rede;
	\item Projeto do conversor CC/CC elevador à saída do retificador junto com o controle MPPT de velocidade da turbina;
	
	\item  Implementação e simulação sistema utilizando o ambiente computacional MATLAB/SIMULINK\texttrademark;
	
	\item Avaliação dos resultados do sistema de acordo com os diferentes cenários de simulação propostos pelo estudo de caso.
\end{itemize}


% 1.3 - Organização do trabalho 
\section{Organização do trabalho}


O trabalho organiza-se em 4 capítulos de acordo com as descrições abaixo:
\begin{itemize}
	\item No capítulo 2 será apresentada a fundamentação teórica do trabalho baseada na revisão bibliográfica. Neste capítulo serão abordados os principais tópicos e as referências utilizadas neste Trabalho de conclusão de Curso.
	\item O capitulo 3 abordará os elementos que compõe o sistema eólico de pequeno porte e como estes serão dimensionados apresentando os parâmetros à serem utilizados nas simulações.
	\item O capítulo 4 expõe os resultados obtidos para os diferentes cenários abordados. Neste capítulo serão apresentados os resultados esperados e os obtidos afim de validar o sistema implementado. além disso, será realizada uma análise da influência do controle MPPT no desempenho do sistema.
	\item Por fim, no Capítulo 5 apresentará as conclusões obtidas com base nos resultados e possíveis cenários para trabalhos futuros. 
\end{itemize}