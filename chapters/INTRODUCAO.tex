Com o constante crescimento da demanda de energia e o cenário de incentivos ao uso de novas alternativas de geração fazem com que as fontes renováveis de energia passem a ter um crescimento cada vez mais constante na matriz energética mundial e brasileira. Com este contexto, passa a ser cada vez mais comum o investimento nessas fontes e o estudo de novas tecnologias relacionadas.

Do ponto de vista da mini e microgeração, as novas tecnologias procuram aumentar a eficiência da geração para melhorar o aproveitamento da energia disponível. No caso da geração eólica de pequeno porte essa tentativa esbarra no aumento significativo do custo total do investimento o que acaba fazendo com que a grande maioria das turbinas de pequeno porte não possuam um sistema MPPT que busca o ponto de máxima potência de geração. Assim, embora possuam um bom retorno os sistemas de pequeno porte acabam desperdiçando boa parte da energia disponível por não possuírem um sistema de busca do ponto ótimo de operação.

Desta forma, este trabalho visa estudar o desempenho de um aerogerador de imã permanente trifásico com e sem controle MPPT ligado à rede elétrica monofásica assim como o ganho de potência com base no controle proposto. Possibilitando assim um estudo de custo benefício entre as duas metodologias estudadas.


Testing


%Área do trabalho

 


% Tema e o contexto, contextualização 

    Estudo de de aerogeradores de imã permanente com e sem MPPT ligados à rede elétrica monofásica.
    
    A energia eólica representa uma das principais alternativas renováveis para geração de
    energia elétrica.
    
    Com a escalada no preço dos combust[iveis fósseis por todo o mundo, o preço médio da energia elétrica têm subido em níveis elevados na última década. Aliado ao aumento da demanda e mundança nos hábitos de consumo da população, estes fatores têm levado governos por todo mundo a incentivar o uso de fontes renováveis de energia. O Brasil, por exemplo, têm regulamentado a troca de créditos de energia para os consumidores que adotarem fontes renováveis como geração distribuída. Mesmo sendo um país com baixa dependência de combustíveis fósseis, o Brasil vem vivenciando um boom no uso da geração distribuída tendo a energia solar e eólica como principais tecnologias utilizadas.
    
    
    
% Delimitação do Tema
    
    Diferentemente da energia solar, a geração eólica de pequeno porte ainda não se apresenta de forma massiva entre as escolhas investimentos em geração distribuída. Por depender da topologia da região, esta forma de geração apresenta-se mais em investimentos de grande porte ocupando vastas áreas. Para baixa potência, 



% Definição do problema

% Objetivos e metodologia 