%% UNIVERSIDADE FEDERAL DA PARAÍBA
%% CENTRO DE ENERGIAS ALTERNATIVAS E RENOVÁVEIS
%% DEPARTAMENTO DE ENGENHARIA ELÉTRICA

%% Criado por Gabriel Luiz De França Sales

%% Arquivo principal TCC


% ----
% Configurações abntex2
% ----
% Cabeçalho
\input{arquives/abntex2_settings.tex}

% Informações da capa

% ---
% Informações de dados para CAPA e FOLHA DE ROSTO
% ---
\titulo{Estudo de aerogeradores e imã permanente com e sem controle MPPT}
\autor{Gabriel Luiz de França Sales}
\local{Brasil}
\data{2019}
\orientador{Rogério Gaspar de Almeida}
\coorientador{}
\instituicao{%
  Universidade Federal da Paraíba -- UFPB
  \par
  Centro de Energias Alternativas e Renováveis
  \par
  Departamento de Engenharia Elétrica}
\tipotrabalho{Trabalho de Conclusão de Curso}
% O preambulo deve conter o tipo do trabalho, o objetivo, 
% o nome da instituição e a área de concentração 
\preambulo{Trabalho de conclusão de Curso apresentado com o fim de obter o título de Engenheiro eletricista o período 2018.2.}
% ---



% Informações do documento
\input{arquives/abntex2_pdfSettings.tex}
% ----

% ----
% Configurações adicionais 
% ----


\renewcommand{\imprimircapa}{%
	\begin{capa}%
		\center
		\includegraphics[width=0.2\textwidth]{figures/logo.jpg} \hfill \includegraphics[width=0.2\textwidth]{figures/cear.png} \\
		\ABNTEXchapterfont\Large UNIVERSIDADE FEDERAL DA PARAÍBA\\CENTRO DE ENERGIAS ALTERNATIVAS E RENOVÁVEIS\\CURSO DE GRADUAÇÂO EM ENGENHARIA ELÉTRICA
		\vspace*{1cm}
		
		{\ABNTEXchapterfont\large\imprimirautor}
		
		\vfill
		\begin{center}
			\ABNTEXchapterfont\bfseries\LARGE\imprimirtitulo
		\end{center}
		\vfill
		
		\large\imprimirlocal
		
		\large\imprimirdata
		
		\vspace*{1cm}
	\end{capa}
}



\usepackage{booktabs,multirow,multicol,colortbl,bigstrut}
\usepackage{longtable}
\usepackage{afterpage}
\usepackage{wrapfig}
\usepackage{amsmath}
\usepackage{tabularx}

% ----


% ----
% Início do documento
% ----
\begin{document}

% Retira espaço extra obsoleto entre as frases.
\frenchspacing 

% ----------------------------------------------------------
% ELEMENTOS PRÉ-TEXTUAIS
% ----------------------------------------------------------
% \pretextual

% ---
% Capa
% ---
\imprimircapa
% ---

% ---
% Folha de rosto
% (o * indica que haverá a ficha bibliográfica)
% ---
\imprimirfolhaderosto*
% ---

% ---
% Inserir a ficha bibliografica
% ---
\input{pretextual/ficha_catalografica.tex}
% ---

% ---
% Inserir errata
% ---
\input{pretextual/errata.tex}
% ---

% ---
% Inserir folha de aprovação
% ---
\input{pretextual/folha_aprovacao.tex}
% ---

% ---
% Dedicatória
% ---
\begin{dedicatoria}
   \vspace*{\fill}
   \centering
   \noindent
   \textit{ Este trabalho é dedicado às crianças adultas que,\\
   quando pequenas, sonharam em se tornar cientistas.} \vspace*{\fill}
\end{dedicatoria}
% ---

% ---
% Agradecimentos
% ---
\input{pretextual/agradecimentos.tex}
% ---

% ---
% Epígrafe
% ---
\begin{epigrafe}
    \vspace*{\fill}
	\begin{flushright}
		\textit{``Não vos amoldeis às estruturas deste mundo, \\
		mas transformai-vos pela renovação da mente, \\
		a fim de distinguir qual é a vontade de Deus: \\
		o que é bom, o que Lhe é agradável, o que é perfeito.\\
		(Bíblia Sagrada, Romanos 12, 2)}
	\end{flushright}
\end{epigrafe}
% ---

% ---
% RESUMOS
% ---
\input{pretextual/resumos.tex}
% ---

% ---
% inserir lista de ilustrações
% ---
\pdfbookmark[0]{\listfigurename}{lof}
\listoffigures*
\cleardoublepage
% ---

% ---
% inserir lista de tabelas
% ---
\pdfbookmark[0]{\listtablename}{lot}
\listoftables*
\cleardoublepage
% ---

% ---
% inserir lista de abreviaturas e siglas
% ---
\begin{siglas}
  \item[ABNT] Associação Brasileira de Normas Técnicas
  \item[abnTeX] ABsurdas Normas para TeX
\end{siglas}
% ---

% ---
% inserir lista de símbolos
% ---
\begin{simbolos}
  \item[$ \Gamma $] Letra grega Gama
  \item[$ \Lambda $] Lambda
  \item[$ \zeta $] Letra grega minúscula zeta
  \item[$ \in $] Pertence
\end{simbolos}
% ---

% ---
% inserir o sumario
% ---
\pdfbookmark[0]{\contentsname}{toc}
\tableofcontents*
\cleardoublepage
% ---



% ----------------------------------------------------------
% ELEMENTOS TEXTUAIS
% ----------------------------------------------------------
\textual

% ----------------------------------------------------------
% Introdução (exemplo de capítulo sem numeração, mas presente no Sumário)
% ----------------------------------------------------------
\chapter*[Introdução]{Introdução}
\addcontentsline{toc}{chapter}{Introdução}
% ----------------------------------------------------------

% Gabriel Luiz De França Sales
% Trabalho de Conclusão de Curso
% Tema: Estudo de Aerogerador Síncrono de Imã Permanente Trifásico com e sem Controle MPPT Conectado à Rede Elétrica Monofásica
%
% Introdução
% Neste arquivo será apresentado o trabalho e sua motivação, além disso serão explicadas as metodologias utilizadas e a estrutura do trabalho

% 1 - Introdução



% 1.1 - Motivação
\section{Motivação}  


% 1.2 - Objetivos
\section{Objetivos}

O objetivo deste trabalho é avaliar o funcionamento de um aerogerador síncrono de imã permanente com e sem controle de velocidade baseado no rastreamento do ponto de máxima potência - MPPT. Desta forma este trabalho visa implementar, a partir de modelagem computacional, o controle de velocidade da turbina assim como o sistema de conexão com a rede elétrica monofásica. Assim, torna-se necessário o estudo dos componentes necessários ao sistema e o dimensionamento adequado dos controladores e filtros utilizados.


% 1.2.1 - Objetivos específicos
\subsection{Objetivos específicos}

Neste trabalho pretende-se desenvolver respectivamente:

\begin{itemize}
	\item Projeto e utilização de um inversor monofásico para a conexão com a rede elétrica juntamente com o controle de tensão do barramento CC à entrada do inversor e o controle de corrente em fase com a tensão da rede;
	\item Projeto do conversor CC/CC elevador à saída do retificador junto com o controle MPPT de velocidade da turbina;
	
	\item  Implementação e simulação sistema utilizando o ambiente computacional MATLAB/SIMULINK\texttrademark;
	
	\item Avaliação dos resultados do sistema de acordo com os diferentes cenários de simulação propostos pelo estudo de caso.
\end{itemize}


% 1.3 - Organização do trabalho 
\section{Organização do trabalho}


O trabalho organiza-se em 4 capítulos de acordo com as descrições abaixo:
\begin{itemize}
	\item No capítulo 2 será apresentada a fundamentação teórica do trabalho baseada na revisão bibliográfica. Neste capítulo serão abordados os principais tópicos e as referências utilizadas neste Trabalho de conclusão de Curso.
	\item O capitulo 3 abordará os elementos que compõe o sistema eólico de pequeno porte e como estes serão dimensionados apresentando os parâmetros à serem utilizados nas simulações.
	\item O capítulo 4 expõe os resultados obtidos para os diferentes cenários abordados. Neste capítulo serão apresentados os resultados esperados e os obtidos afim de validar o sistema implementado. além disso, será realizada uma análise da influência do controle MPPT no desempenho do sistema.
	\item Por fim, no Capítulo 5 apresentará as conclusões obtidas com base nos resultados e possíveis cenários para trabalhos futuros. 
\end{itemize}



% ----------------------------------------------------------
% PARTE
% ----------------------------------------------------------
\part{Preparação da pesquisa}
% ----------------------------------------------------------

% ---
% Capitulo com exemplos de comandos inseridos de arquivo externo 
% ---
\include{abntex2-modelo-include-comandos}
% ---

% ----------------------------------------------------------
% PARTE
% ----------------------------------------------------------
\part{Referenciais teóricos}
% ----------------------------------------------------------

% ---
% Capitulo de revisão de literatura
% ---
\chapter{Lorem ipsum dolor sit amet}
% ---

% ---
\section{Aliquam vestibulum fringilla lorem}
% ---

\lipsum[1]

\lipsum[2-3]

% ----------------------------------------------------------
% PARTE
% ----------------------------------------------------------
\part{Resultados}
% ----------------------------------------------------------

% ---
% primeiro capitulo de Resultados
% ---
\chapter{Lectus lobortis condimentum}
% ---

% ---
\section{Vestibulum ante ipsum primis in faucibus orci luctus et ultrices
posuere cubilia Curae}
% ---

\lipsum[21-22]

% ---
% segundo capitulo de Resultados
% ---
\chapter{Nam sed tellus sit amet lectus urna ullamcorper tristique interdum
elementum}
% ---

% ---
\section{Pellentesque sit amet pede ac sem eleifend consectetuer}
% ---

\lipsum[24]

% ----------------------------------------------------------
% Finaliza a parte no bookmark do PDF
% para que se inicie o bookmark na raiz
% e adiciona espaço de parte no Sumário
% ----------------------------------------------------------
%\phantompart

% ---
% Conclusão (outro exemplo de capítulo sem numeração e presente no sumário)
% ---
\chapter*[Conclusão]{Conclusão}
\addcontentsline{toc}{chapter}{Conclusão}
% ---

\lipsum[31-33]

% ----------------------------------------------------------
% ELEMENTOS PÓS-TEXTUAIS
% ----------------------------------------------------------
\postextual
% ----------------------------------------------------------

% ----------------------------------------------------------
% Referências bibliográficas
% ----------------------------------------------------------
\bibliography{abntex2-modelo-references}

% ----------------------------------------------------------
% Glossário
% ----------------------------------------------------------
%
% Consulte o manual da classe abntex2 para orientações sobre o glossário.
%
%\glossary

% ----------------------------------------------------------
% Apêndices
% ----------------------------------------------------------

% ---
% Inicia os apêndices
% ---
\begin{apendicesenv}

% Imprime uma página indicando o início dos apêndices
\partapendices

% ----------------------------------------------------------
\chapter{Quisque libero justo}
% ----------------------------------------------------------

\lipsum[50]

% ----------------------------------------------------------
\chapter{Nullam elementum urna vel imperdiet sodales elit ipsum pharetra ligula
ac pretium ante justo a nulla curabitur tristique arcu eu metus}
% ----------------------------------------------------------
\lipsum[55-57]

\end{apendicesenv}
% ---


% ----------------------------------------------------------
% Anexos
% ----------------------------------------------------------

% ---
% Inicia os anexos
% ---
\begin{anexosenv}

% Imprime uma página indicando o início dos anexos
\partanexos

% ---
\chapter{Morbi ultrices rutrum lorem.}
% ---
\lipsum[30]

% ---
\chapter{Cras non urna sed feugiat cum sociis natoque penatibus et magnis dis
parturient montes nascetur ridiculus mus}
% ---

\lipsum[31]

% ---
\chapter{Fusce facilisis lacinia dui}
% ---

\lipsum[32]

\end{anexosenv}

%---------------------------------------------------------------------
% INDICE REMISSIVO
%---------------------------------------------------------------------
%\phantompart
\printindex
%---------------------------------------------------------------------

\end{document}
